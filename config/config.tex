%**************************************************************
% file contenente le impostazioni della tesi
%**************************************************************

%**************************************************************
% Frontespizio
%**************************************************************
\newcommand{\myName}{Andrea Mantovani}                          % autore
\newcommand{\myTitle}{Realizzazione di un modulo NER grafico per riconoscere "annotazioni" all'interno di recensioni}
\newcommand{\myDegree}{Tesi di laurea triennale}                % tipo di tesi
\newcommand{\myUni}{Università degli Studi di Padova}           % università
\newcommand{\myFaculty}{Corso di Laurea in Informatica}         % facoltà
\newcommand{\myDepartment}{Dipartimento di Matematica}          % dipartimento
\newcommand{\myProf}{Gilberto Filè}                             % relatore
\newcommand{\myLocation}{Padova}                                % dove
\newcommand{\myAA}{2015-2016}                                   % anno accademico
\newcommand{\myTime}{Sett 2016}                                 % quando

%**************************************************************
% Impostazioni di impaginazione
% see: http://wwwcdf.pd.infn.it/AppuntiLinux/a2547.htm
%**************************************************************

\setlength{\parindent}{14pt}   % larghezza rientro della prima riga
\setlength{\parskip}{0pt}   % distanza tra i paragrafi


%**************************************************************
% Impostazioni di biblatex
%**************************************************************
\bibliography{res/bibliography} % database di biblatex

\defbibheading{bibliography}
{
    \cleardoublepage
    \phantomsection
    \addcontentsline{toc}{chapter}{\bibname}
    \chapter*{\bibname\markboth{\bibname}{\bibname}}
}

\setlength\bibitemsep{1.5\itemsep} % spazio tra entry

\DeclareBibliographyCategory{opere}
\DeclareBibliographyCategory{web}
\DeclareBibliographyCategory{book}

\addtocategory{web}{site:agile-manifesto}
\addtocategory{web}{site:ing-software}
\addtocategory{web}{site:martin-flower}
\addtocategory{web}{site:wiki-scripting-language}
\addtocategory{web}{site:nltk}
\addtocategory{web}{site:wiki-separation-concerns}
\addtocategory{web}{site:wiki-representational-state-transfer}
\addtocategory{web}{site:wiki-web-service}
\addtocategory{web}{site:web-service-architecture}
\addtocategory{web}{site:treccani-interoperabilita}
\addtocategory{web}{site:wiki-data-access-object}
\addtocategory{web}{site:node-js}
\addtocategory{web}{site:trademark-policy-pdf}
\addtocategory{web}{site:mongodb-brand-resources}
\addtocategory{web}{site:wiki-unix-philosophy}
\addtocategory{web}{site:basics-of-the-unix-philosophy}
\addtocategory{web}{site:wiki-dynamic-web-page}
\addtocategory{web}{site:wiki-single-page-application}
\addtocategory{web}{site:what-is-docker}
\addtocategory{web}{site:wiki-virtual-machine}
\addtocategory{web}{site:lingpipe}
\addtocategory{web}{site:wiki-linguistica-computazionale}
\addtocategory{web}{site:protractor}
\addtocategory{web}{site:mongoose}
\addtocategory{web}{site:mongoose-model}
\addtocategory{web}{site:architettura-elaboratori}
\addtocategory{book}{software-engineering}

\defbibheading{web}{\section*{Siti Web consultati}}
\defbibheading{book}{\section*{Libri consultati}}

%**************************************************************
% Impostazioni di caption
%**************************************************************
\captionsetup{
    tableposition=top,
    figureposition=bottom,
    font=small,
    format=hang,
    labelfont=bf
}

%**************************************************************
% Impostazioni di glossaries
%**************************************************************
\newglossaryentry{agile} {
name=Agile,
description={
Insieme di principi per lo sviluppo software sotto i quali requisiti e
soluzioni evolvono trammite gli sforzi collettivi del gruppo auto-organizzato
polifunzionale.
}
}

\newglossaryentry{application-programming-interface} {
name=Application Programming Interface,
description={
Insieme di procedure disponibili al programmatore, di solito raggruppate a
formare un set di strumenti specifici per l’espletamento di un determinato
compito all’interno di un certo programma.
}
}

\newglossaryentry{backend-as-a-service} {
name=Backend as a Service,
description={
Modello per fornire agli sviluppatori di applicazioni web di collegare le loro
applicazioni ad un \glslink{back-end} tramite API.
}
}

\newglossaryentry{back-end} {
name=back end,
description={
Termine inglese che indica la parte di un sistema software che si occupa della
memorizzazione e del recupero dei dati.
}
}

\newglossaryentry{business-to-business} {
name=Business-to-business,
description={
Locuzione utilizzata per descrivere le transizioni commerciali tra imprese.
}
}

\newglossaryentry{contratto} {
name=contratto,
description={
Nel contesto dello sviluppo software, il contratto è l'insieme delle formali,
precise e verificabili specifiche di un'interfaccia per un componente software.
}
}

\newglossaryentry{version-control-system} {
name=version control system,
description={
Software per la gestione dei cambiamenti all'interno di una collezzione di
elementi software.
}
}

\newglossaryentry{dashboard} {
name=dashboard,
description={
Interfaccia utente dove vengono esposte le informazioni più rilevanti e di
frequente consultazione in base all'argomento.
}
}

\newglossaryentry{data-access-object} {
name=data access object,
description={
Un oggetto che provvede un'interfaccia per specifiche operazioni sul database
senza esporne i dettagli.
}
}

\newglossaryentry{database-management-system} {
name=Database Management System,
description={
Un sistema software progettato per consentire la creazione, la manipolazione
e l'interrogazione efficiente di database.
}
}

\newglossaryentry{dynamic-web-page} {
name=dynamic web page,
description={
Una pagina web la cui costruzione viene controllata da un'applicazione lato
server in base agli input dell'utente.
}
}

\newglossaryentry{front-end} {
name=front end,
description={
Parte visibile all’utente e con cui egli può interagire; nella sua accezione più
generale, è responsabile dell’acquisizione dei dati di ingresso e della loro
elaborazione con modalità conformi a specifiche predefinite e invarianti, tali
da renderli utilizzabili dal back end.
}
}

\newglossaryentry{end-to-end-test} {
name=end-to-end test,
description={
Categoria di test dove si vanno a verificare che il comportamento delle
interfacce dei componenti, soggetti ai test, sia conforme alla loro specifica.
}
}

\newglossaryentry{event-driven} {
name=event driven,
description={
Paradigma di programmazione il quale il flusso di controllo dell'applicazione è
determinato dagli eventi (pressione del mouse, sensore, ecc...).
}
}

\newglossaryentry{interoperabilita} {
name=interoperabilità,
description={
Capacità di due o più sistemi, reti, mezzi, applicazioni o componenti, di
scambiare informazioni tra loro e di essere poi in grado di utilizzarle.
}
}

\newglossaryentry{javascript_object_notation} {
name=JavaScript Object Notation,
description={
Formato adatto all’interscambio di dati fra applicazioni client-server. La
semplicità di JSON ne ha decretato un rapido utilizzo specialmente nella
programmazione in AJAX (Asynchronous JavaScript and XML). Il suo uso tramite
JavaScript è particolarmente semplice e questo fatto lo ha reso velocemente
molto popolare.
}
}

\newglossaryentry{linguistica-computazionale} {
name=linguistica computazionale,
description={
La linguistica computazionale si concentra sullo sviluppo di formalismi
descrittivi del funzionamento di una lingua naturale, tali che si possano
trasformare in programmi eseguibili dai computer.
}
}

\newglossaryentry{macchina-virtuale} {
name=macchina virtuale,
description={
Emulazione di una determinata macchina reale.
},
plural=macchine virtuali
}

\newglossaryentry{named_entity_recognition} {
name=Named Entity Recognition,
description={
Processo di ricerca di tutte le menzioni riguardo specifiche cose in un testo.
}
}

\newglossaryentry{pipeline} {
name=pipeline,
description={
Catena di unità d'elaborazione dove l'output di ogni unità è l'input di quella
successiva.
}
}

\newglossaryentry{random-access-memory} {
name=Random Access Memory,
description={
Memoria di tipo volatile che permette l'accesso in tempo costante ad ogni suo
indirizzo di memoria.
}
}

\newglossaryentry{refactoring} {
name=refactoring,
description={
Tecnica controllata per migliorare la progettazione di un codice base già
esistente.
}
}

\newglossaryentry{representational_state_transfer} {
name=representational state transfer,
description={
Insieme di principi di architetture di rete, i quali delineano come le risorse
sono definite e indirizzate. In particolare prevede che lo stato
dell’applicazione e le funzionalità siano divisi in risorse web, che ogni
risorsa sia unica e indirizzabile usando sintassi universale per uso nei link
ipertestuali e che tutte le risorse siano condivise come interfaccia uniforme
per il trasferimento di stato tra client e risorse (attraverso un insieme
vincolato di operazioni ben definite e un protocollo stateless).
}
}

\newglossaryentry{script} {
name=script,
description={
Programmi il cui compito è automatizzare l'esecuzione di compiti che in
alternativa dovrebbero essere eseguiti da operatori umani.
}
}

\newglossaryentry{scrum} {
name=Scrum,
description={
SDLC iterativo-incrementale concepito per rilasci brevi e programmati del
software. L'obiettivo di Scrum è essere flessibile al cambio dei requisiti
imposti dal cliente, in modo da poter soddisfare al meglio mantenendo un
controllo sulla qualità e le tempistiche del progetto.
}
}

\newglossaryentry{separation_of_concerns} {
name=Separation of concerns,
description={
Un principio di progettazione software per separare un programma in varie
sezioni.
}
}

\newglossaryentry{single-page-application} {
name=single-page application,
description={
Applicazione web, o sito web, composta da una sola pagina le cui risorse vengono
caricate dinamicante trammite una continua interazione con il server.
}
}

\newglossaryentry{sprint} {
name=sprint,
description={
Unità base della fase operativa nel modello di sviluppo \glslink{scrum} e
corrisponde ad una iterazione.
}
}

\newglossaryentry{stub} {
name=stub,
description={
Frammento di codice passivo e fittizio usato per simulare il comportamento di
codice esistente o per sostituire codice non ancora implementato.
}
}

\newglossaryentry{verifica} {
name=verifica,
description={
Attività volta alla ricerca di consistenza, correttezza e completezza.
}
}

\newglossaryentry{web-service} {
name=Web Service,
description={
Un sistema software progettato per supportare l'interoperabilità tra elaboratori
su una rete.
}
}

\makeglossaries

\newacronym[see={[Glossary:]{business-to-business}}]{B2B}{B2B}{Business-to-business}
\newacronym[see={[Glossary:]{named_entity_recognition}}]{NER}{NER}{Named Entity Recognition}
\newacronym[see={[Glossary:]{javascript_object_notation}}]{JSON}{JSON}{JavaScript Object Notation}
\newacronym[see={[Glossary:]{representational_state_transfer}}]{REST}{REST}{representational state transfer}
\newacronym[see={[Glossary:]{backend-as-a-service}}]{BaaS}{BaaS}{Backend as a Service}
\newacronym[see={[Glossary:]{application-programming-interface}}]{API}{API}{Application Programming Interface}
\newacronym[see={[Glossary:]{database-management-system}}]{DBMS}{DBMS}{Database Management System}
\newacronym[see={[Glossary:]{data-access-object}}]{DAO}{DAO}{data access object}
\newacronym[see={[Glossary:]{random-access-memory}}]{RAM}{RAM}{Random Access Memory}
\newacronym[see={[Glossary:]{control-version-system}}]{CVS}{CVS}{control version system}
\newacronym[see={[Glossary:]{end-to-end-test}}]{E2ET}{E2E test}{end-to-end test}
 % database di termini
\makeglossaries

%**************************************************************
% Impostazioni di graphicx
%**************************************************************
\graphicspath{{res/img/}} % cartella dove sono riposte le immagini


%**************************************************************
% Impostazioni di hyperref
%**************************************************************
\hypersetup{
    %hyperfootnotes=false,
    %pdfpagelabels,
    %draft,	% = elimina tutti i link (utile per stampe in bianco e nero)
    colorlinks=true,
    linktocpage=true,
    pdfstartpage=1,
    pdfstartview=FitV,
    % decommenta la riga seguente per avere link in nero (per esempio per la stampa in bianco e nero)
    %colorlinks=false, linktocpage=false, pdfborder={0 0 0}, pdfstartpage=1, pdfstartview=FitV,
    breaklinks=true,
    pdfpagemode=UseNone,
    pageanchor=true,
    pdfpagemode=UseOutlines,
    plainpages=false,
    bookmarksnumbered,
    bookmarksopen=true,
    bookmarksopenlevel=1,
    hypertexnames=true,
    pdfhighlight=/O,
    %nesting=true,
    %frenchlinks,
    urlcolor=webbrown,
    linkcolor=RoyalBlue,
    citecolor=webgreen,
    %pagecolor=RoyalBlue,
    %urlcolor=Black, linkcolor=Black, citecolor=Black, %pagecolor=Black,
    pdftitle={\myTitle},
    pdfauthor={\textcopyright\ \myName, \myUni, \myFaculty},
    pdfsubject={},
    pdfkeywords={},
    pdfcreator={pdfLaTeX},
    pdfproducer={LaTeX}
}

%**************************************************************
% Impostazioni di itemize
%**************************************************************
\renewcommand{\labelitemi}{$\ast$}

%\renewcommand{\labelitemi}{$\bullet$}
%\renewcommand{\labelitemii}{$\cdot$}
%\renewcommand{\labelitemiii}{$\diamond$}
%\renewcommand{\labelitemiv}{$\ast$}


%**************************************************************
% Impostazioni di listings
%**************************************************************
\lstset{
    language=[LaTeX]Tex,%C++,
    keywordstyle=\color{RoyalBlue}, %\bfseries,
    basicstyle=\small\ttfamily,
    %identifierstyle=\color{NavyBlue},
    commentstyle=\color{Green}\ttfamily,
    stringstyle=\rmfamily,
    numbers=none, %left,%
    numberstyle=\scriptsize, %\tiny
    stepnumber=5,
    numbersep=8pt,
    showstringspaces=false,
    breaklines=true,
    frameround=ftff,
    frame=single
}


%**************************************************************
% Impostazioni di xcolor
%**************************************************************
\definecolor{webgreen}{rgb}{0,.5,0}
\definecolor{webbrown}{rgb}{.6,0,0}


%**************************************************************
% Altro
%**************************************************************

\newcommand{\omissis}{[\dots\negthinspace]} % produce [...]

% eccezioni all'algoritmo di sillabazione
\hyphenation
{
    ma-cro-istru-zio-ne
    gi-ral-din
}

\newcommand{\sectionname}{sezione}
\addto\captionsitalian{\renewcommand{\figurename}{figura}
                       \renewcommand{\tablename}{tabella}}

\newcommand{\glsfirstoccur}{\ap{{[g]}}}

\newcommand{\intro}[1]{\emph{\textsf{#1}}}

%**************************************************************
% Environment per ``rischi''
%**************************************************************
\newcounter{riskcounter}                % define a counter
\setcounter{riskcounter}{0}             % set the counter to some initial value

%%%% Parameters
% #1: Title
\newenvironment{risk}[1]{
    \refstepcounter{riskcounter}        % increment counter
    \par \noindent                      % start new paragraph
    \textbf{\arabic{riskcounter}. #1}   % display the title before the
                                        % content of the environment is displayed
}{
    \par\medskip
}

\newcommand{\riskname}{Rischio}

\newcommand{\riskdescription}[1]{\textbf{\\Descrizione:} #1.}

\newcommand{\risksolution}[1]{\textbf{\\Soluzione:} #1.}

%**************************************************************
% Environment per ``use case''
%**************************************************************
\newcounter{usecasecounter}             % define a counter
\setcounter{usecasecounter}{0}          % set the counter to some initial value

%%%% Parameters
% #1: ID
% #2: Nome
\newenvironment{usecase}[2]{
    \renewcommand{\theusecasecounter}{\usecasename #1}  % this is where the display of
                                                        % the counter is overwritten/modified
    \refstepcounter{usecasecounter}             % increment counter
    \vspace{10pt}
    \par \noindent                              % start new paragraph
    {\large \textbf{\usecasename #1: #2}}       % display the title before the
                                                % content of the environment is displayed
    \medskip
}{
    \medskip
}

\newcommand{\usecasename}{UC}

\newcommand{\usecaseactors}[1]{\textbf{\\Attori Principali:} #1. \vspace{4pt}}
\newcommand{\usecasepre}[1]{\textbf{\\Precondizioni:} #1. \vspace{4pt}}
\newcommand{\usecasedesc}[1]{\textbf{\\Descrizione:} #1. \vspace{4pt}}
\newcommand{\usecasepost}[1]{\textbf{\\Postcondizioni:} #1. \vspace{4pt}}
\newcommand{\usecasealt}[1]{\textbf{\\Scenario Alternativo:} #1. \vspace{4pt}}

%**************************************************************
% Environment per ``namespace description''
%**************************************************************

\newenvironment{namespacedesc}{
    \vspace{10pt}
    \par \noindent                              % start new paragraph
    \begin{description}
}{
    \end{description}
    \medskip
}

\newcommand{\classdesc}[2]{\item[\textbf{#1:}] #2}
