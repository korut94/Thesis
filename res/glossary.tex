\newglossaryentry{agile} {
name=Agile,
description={
Insieme di principi per lo sviluppo software sotto i quali requisiti e
soluzioni evolvono trammite gli sforzi collettivi del gruppo auto-organizzato
polifunzionale.
}
}

\newglossaryentry{business-to-business} {
name=Business-to-business,
description={
Locuzione utilizzata per descrivere le transizioni commerciali tra imprese.
}
}

\newglossaryentry{dashboard} {
name=dashboard,
description={
Interfaccia utente dove vengono esposte le informazioni più rilevanti e di
frequente consultazione in base all'argomento.
}
}

\newglossaryentry{refactoring} {
name=refactoring,
description={
Tecnica controllata per migliorare la progettazione di un codice base già
esistente.
}
}

\newglossaryentry{script} {
name=script,
description={
Programmi il cui compito è automatizzare l'esecuzione di compiti che in
alternativa dovrebbero essere eseguiti da operatori umani.
}
}

\newglossaryentry{scrum} {
name=Scrum,
description={
SDLC iterativo-incrementale concepito per rilasci brevi e programmati del
software. L'obiettivo di Scrum è essere flessibile al cambio dei requisiti
imposti dal cliente, in modo da poter soddisfare al meglio mantenendo un
controllo sulla qualità e le tempistiche del progetto.
}
}

\newglossaryentry{sprint} {
name=sprint,
description={
Unità base della fase operativa nel modello di sviluppo \glslink{scrum} e
corrisponde ad una iterazione.
}
}

\newglossaryentry{verifica} {
name=verifica,
description={
Attività volta alla ricerca di consistenza, correttezza e completezza.
}
}

\makeglossaries

\newacronym[see={[Glossary:]{business-to-business}}]{B2B}{B2B}{Business-to-busin
ss}
