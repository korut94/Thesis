\newglossaryentry{agile} {
name=Agile,
description={
Insieme di principi per lo sviluppo software sotto i quali requisiti e
soluzioni evolvono trammite gli sforzi collettivi del gruppo auto-organizzato
polifunzionale.
}
}

\newglossaryentry{application-programming-interface} {
name=Application Programming Interface,
description={
Insieme di procedure disponibili al programmatore, di solito raggruppate a
formare un set di strumenti specifici per l’espletamento di un determinato
compito all’interno di un certo programma.
}
}

\newglossaryentry{backend-as-a-service} {
name=Backend as a Service,
description={
Modello per fornire agli sviluppatori di applicazioni web di collegare le loro
applicazioni ad un \glslink{back-end} tramite API.
}
}

\newglossaryentry{back-end} {
name=back end,
description={
Termine inglese che indica la parte di un sistema software che si occupa della
memorizzazione e del recupero dei dati.
}
}

\newglossaryentry{business-to-business} {
name=Business-to-business,
description={
Locuzione utilizzata per descrivere le transizioni commerciali tra imprese.
}
}

\newglossaryentry{contratto} {
name=contratto,
description={
Nel contesto dello sviluppo software, il contratto è l'insieme delle formali,
precise e verificabili specifiche di un'interfaccia per un componente software.
}
}

\newglossaryentry{dashboard} {
name=dashboard,
description={
Interfaccia utente dove vengono esposte le informazioni più rilevanti e di
frequente consultazione in base all'argomento.
}
}

\newglossaryentry{data-access-object} {
name=data access object,
description={
Un oggetto che provvede un'interfaccia per specifiche operazioni sul database
senza esporne i dettagli.
}
}

\newglossaryentry{database-management-system} {
name=Database Management System,
description={
Un sistema software progettato per consentire la creazione, la manipolazione
e l'interrogazione efficiente di database.
}
}

\newglossaryentry{front-end} {
name=front end,
description={
Parte visibile all’utente e con cui egli può interagire; nella sua accezione più
generale, è responsabile dell’acquisizione dei dati di ingresso e della loro
elaborazione con modalità conformi a specifiche predefinite e invarianti, tali
da renderli utilizzabili dal back end.
}
}

\newglossaryentry{event-driven} {
name=event driven,
description={
Paradigma di programmazione il quale il flusso di controllo dell'applicazione è
determinato dagli eventi (pressione del mouse, sensore, ecc...).
}
}

\newglossaryentry{interoperabilita} {
name=interoperabilità,
description={
Capacità di due o più sistemi, reti, mezzi, applicazioni o componenti, di
scambiare informazioni tra loro e di essere poi in grado di utilizzarle.
}
}

\newglossaryentry{javascript_object_notation} {
name=JavaScript Object Notation,
description={
Formato adatto all’interscambio di dati fra applicazioni client-server. La
semplicità di JSON ne ha decretato un rapido utilizzo specialmente nella
programmazione in AJAX (Asynchronous JavaScript and XML). Il suo uso tramite
JavaScript è particolarmente semplice e questo fatto lo ha reso velocemente
molto popolare.
}
}

\newglossaryentry{named_entity_recognition} {
name=Named Entity Recognition,
description={
Sistema di riconoscimento di tutte le menzioni testuali dei
\textit{Named Entity}, individuando la sua posizione ed il suo tipo.
}
}

\newglossaryentry{refactoring} {
name=refactoring,
description={
Tecnica controllata per migliorare la progettazione di un codice base già
esistente.
}
}

\newglossaryentry{representational_state_transfer} {
name=representational state transfer,
description={
Insieme di principi di architetture di rete, i quali delineano come le risorse
sono definite e indirizzate. In particolare prevede che lo stato
dell’applicazione e le funzionalità siano divisi in risorse web, che ogni
risorsa sia unica e indirizzabile usando sintassi universale per uso nei link
ipertestuali e che tutte le risorse siano condivise come interfaccia uniforme
per il trasferimento di stato tra client e risorse (attraverso un insieme
vincolato di operazioni ben definite e un protocollo stateless).
}
}

\newglossaryentry{script} {
name=script,
description={
Programmi il cui compito è automatizzare l'esecuzione di compiti che in
alternativa dovrebbero essere eseguiti da operatori umani.
}
}

\newglossaryentry{scrum} {
name=Scrum,
description={
SDLC iterativo-incrementale concepito per rilasci brevi e programmati del
software. L'obiettivo di Scrum è essere flessibile al cambio dei requisiti
imposti dal cliente, in modo da poter soddisfare al meglio mantenendo un
controllo sulla qualità e le tempistiche del progetto.
}
}

\newglossaryentry{separation_of_concerns} {
name=Separation of concerns,
description={
Un principio di progettazione software per separare un programma in varie
sezioni.
}
}

\newglossaryentry{sprint} {
name=sprint,
description={
Unità base della fase operativa nel modello di sviluppo \glslink{scrum} e
corrisponde ad una iterazione.
}
}

\newglossaryentry{web-service} {
name=Web Service,
description={
Un sistema software progettato per supportare l'interoperabilità tra elaboratori
su una rete.
}
}

\newglossaryentry{verifica} {
name=verifica,
description={
Attività volta alla ricerca di consistenza, correttezza e completezza.
}
}

\makeglossaries

\newacronym[see={[Glossary:]{business-to-business}}]{B2B}{B2B}{Business-to-business}
\newacronym[see={[Glossary:]{named_entity_recognition}}]{NER}{NER}{Named Entity Recognition}
\newacronym[see={[Glossary:]{javascript_object_notation}}]{JSON}{JSON}{JavaScript Object Notation}
\newacronym[see={[Glossary:]{representational_state_transfer}}]{REST}{REST}{representational state transfer}
\newacronym[see={[Glossary:]{backend-as-a-service}}]{BaaS}{BaaS}{Backend as a Service}
\newacronym[see={[Glossary:]{application-programming-interface}}]{API}{API}{Application Programming Interface}
\newacronym[see={[Glossary:]{database-management-system}}]{DBMS}{DBMS}{Database Management System}
\newacronym[see={[Glossary:]{data-access-object}}]{DAO}{DAO}{data access object}
