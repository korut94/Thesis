\newglossaryentry{agile} {
name=Agile,
description={
Insieme di principi per lo sviluppo software sotto i quali requisiti e
soluzioni evolvono trammite gli sforzi collettivi del gruppo auto-organizzato
polifunzionale.
}
}

\newglossaryentry{application-programming-interface} {
name=Application Programming Interface,
description={
Insieme di procedure disponibili al programmatore, di solito raggruppate a
formare un set di strumenti specifici per l’espletamento di un determinato
compito all’interno di un certo programma.
}
}

\newglossaryentry{backend-as-a-service} {
name=Backend as a Service,
description={
Modello per fornire agli sviluppatori di applicazioni web di collegare le loro
applicazioni ad un \glslink{back-end} tramite API.
}
}

\newglossaryentry{back-end} {
name=back end,
description={
Termine inglese che indica la parte di un sistema software che si occupa della
memorizzazione e del recupero dei dati.
}
}

\newglossaryentry{business-to-business} {
name=Business-to-business,
description={
Locuzione utilizzata per descrivere le transizioni commerciali tra imprese.
}
}

\newglossaryentry{contratto} {
name=contratto,
description={
Nel contesto dello sviluppo software, il contratto è l'insieme delle formali,
precise e verificabili specifiche di un'interfaccia per un componente software.
}
}

\newglossaryentry{cascading-style-sheets} {
name=Cascading Style Sheet,
description={
Linguaggio usato per definire la formattazione di documenti HTML, XHTML e
XML ad esempio i siti web e relative pagine web. Le regole per comporre il CSS
sono contenute in un insieme di direttive (Recommendations) emanate a partire
dal 1996 dal W3C. L’introduzione del CSS si è resa necessaria per separare i
contenuti delle pagine HTML dalla loro formattazione e permettere una
programmazione più chiara e facile da utilizzare, sia per gli autori delle
pagine stesse sia per gli utenti, garantendo contemporaneamente anche il
riutilizzo di codice ed una sua più facile manutenzione.
}
}

\newglossaryentry{check} {
name=check,
description={
I file al quale è stato effettuato il \textit{check} sono tracciati dal
controllo di versione.
}
}

\newglossaryentry{version-control-system} {
name=version control system,
description={
Software per la gestione dei cambiamenti all'interno di una collezzione di
elementi software.
}
}

\newglossaryentry{dashboard} {
name=dashboard,
description={
Interfaccia utente dove vengono esposte le informazioni più rilevanti e di
frequente consultazione in base all'argomento.
}
}

\newglossaryentry{dependency-injection} {
name=Dependency Injection,
description={
Design pattern che accorda come le componenti di un sistema software si
procurino le proprie dipendenze.
}
}

\newglossaryentry{data-access-object} {
name=data access object,
description={
Un oggetto che provvede un'interfaccia per specifiche operazioni sul database
senza esporne i dettagli.
}
}

\newglossaryentry{database-management-system} {
name=Database Management System,
description={
Un sistema software progettato per consentire la creazione, la manipolazione
e l'interrogazione efficiente di database.
}
}

\newglossaryentry{document-object-model} {
name=Document Object Model,
description={
Insieme di \glsref{API} che trattano documenti HTML, XHTML, o XML come una
struttura ad albero nella quale ogni nodo rappresenta una parte del documento.
Gli oggetti possono essere manipolati e ogni cambiamento visibile può essere
riflettuto nella visualizzazione del documento.
}
}

\newglossaryentry{dynamic-web-page} {
name=dynamic web page,
description={
Una pagina web la cui costruzione viene controllata da un'applicazione lato
server in base agli input dell'utente.
}
}

\newglossaryentry{front-end} {
name=front end,
description={
Parte visibile all’utente e con cui egli può interagire; nella sua accezione più
generale, è responsabile dell’acquisizione dei dati di ingresso e della loro
elaborazione con modalità conformi a specifiche predefinite e invarianti, tali
da renderli utilizzabili dal back end.
}
}

\newglossaryentry{end-to-end-test} {
name=end-to-end test,
description={
Categoria di test dove si vanno a verificare che il comportamento delle
interfacce dei componenti, soggetti ai test, sia conforme alla loro specifica.
}
}

\newglossaryentry{event-driven} {
name=event driven,
description={
Paradigma di programmazione il quale il flusso di controllo dell'applicazione è
determinato dagli eventi (pressione del mouse, sensore, ecc...).
}
}

\newglossaryentry{interoperabilita} {
name=interoperabilità,
description={
Capacità di due o più sistemi, reti, mezzi, applicazioni o componenti, di
scambiare informazioni tra loro e di essere poi in grado di utilizzarle.
}
}

\newglossaryentry{javascript_object_notation} {
name=JavaScript Object Notation,
description={
Formato adatto all’interscambio di dati fra applicazioni client-server. La
semplicità di JSON ne ha decretato un rapido utilizzo specialmente nella
programmazione in AJAX (Asynchronous JavaScript and XML). Il suo uso tramite
JavaScript è particolarmente semplice e questo fatto lo ha reso velocemente
molto popolare.
}
}

\newglossaryentry{linguistica-computazionale} {
name=linguistica computazionale,
description={
La linguistica computazionale si concentra sullo sviluppo di formalismi
descrittivi del funzionamento di una lingua naturale, tali che si possano
trasformare in programmi eseguibili dai computer.
}
}

\newglossaryentry{macchina-virtuale} {
name=macchina virtuale,
description={
Emulazione di una determinata macchina reale.
},
plural=macchine virtuali
}

\newglossaryentry{named_entity_recognition} {
name=Named Entity Recognition,
description={
Processo di ricerca di tutte le menzioni riguardo specifiche cose in un testo.
}
}

\newglossaryentry{pipeline} {
name=pipeline,
description={
Catena di unità d'elaborazione dove l'output di ogni unità è l'input di quella
successiva.
}
}

\newglossaryentry{random-access-memory} {
name=Random Access Memory,
description={
Memoria di tipo volatile che permette l'accesso in tempo costante ad ogni suo
indirizzo di memoria.
}
}

\newglossaryentry{refactoring} {
name=refactoring,
description={
Tecnica controllata per migliorare la progettazione di un codice base già
esistente.
}
}

\newglossaryentry{representational_state_transfer} {
name=representational state transfer,
description={
Insieme di principi di architetture di rete, i quali delineano come le risorse
sono definite e indirizzate. In particolare prevede che lo stato
dell’applicazione e le funzionalità siano divisi in risorse web, che ogni
risorsa sia unica e indirizzabile usando sintassi universale per uso nei link
ipertestuali e che tutte le risorse siano condivise come interfaccia uniforme
per il trasferimento di stato tra client e risorse (attraverso un insieme
vincolato di operazioni ben definite e un protocollo stateless).
}
}

\newglossaryentry{script} {
name=script,
description={
Programmi il cui compito è automatizzare l'esecuzione di compiti che in
alternativa dovrebbero essere eseguiti da operatori umani.
}
}

\newglossaryentry{scrum} {
name=Scrum,
description={
SDLC iterativo-incrementale concepito per rilasci brevi e programmati del
software. L'obiettivo di Scrum è essere flessibile al cambio dei requisiti
imposti dal cliente, in modo da poter soddisfare al meglio mantenendo un
controllo sulla qualità e le tempistiche del progetto.
}
}

\newglossaryentry{separation_of_concerns} {
name=Separation of concerns,
description={
Un principio di progettazione software per separare un programma in varie
sezioni.
}
}

\newglossaryentry{single-page-application} {
name=single-page application,
description={
Applicazione web, o sito web, composta da una sola pagina le cui risorse vengono
caricate dinamicante trammite una continua interazione con il server.
}
}

\newglossaryentry{sprint} {
name=sprint,
description={
Unità base della fase operativa nel modello di sviluppo \glslink{scrum} e
corrisponde ad una iterazione.
}
}

\newglossaryentry{stub} {
name=stub,
description={
Frammento di codice passivo e fittizio usato per simulare il comportamento di
codice esistente o per sostituire codice non ancora implementato.
}
}

\newglossaryentry{verifica} {
name=verifica,
description={
Attività volta alla ricerca di consistenza, correttezza e completezza.
}
}

\newglossaryentry{web-service} {
name=Web Service,
description={
Un sistema software progettato per supportare l'interoperabilità tra elaboratori
su una rete.
}
}

\makeglossaries

\newacronym[see={[Glossary:]{business-to-business}}]{B2B}{B2B}{Business-to-business}
\newacronym[see={[Glossary:]{named_entity_recognition}}]{NER}{NER}{Named Entity Recognition}
\newacronym[see={[Glossary:]{javascript_object_notation}}]{JSON}{JSON}{JavaScript Object Notation}
\newacronym[see={[Glossary:]{representational_state_transfer}}]{REST}{REST}{representational state transfer}
\newacronym[see={[Glossary:]{backend-as-a-service}}]{BaaS}{BaaS}{Backend as a Service}
\newacronym[see={[Glossary:]{application-programming-interface}}]{API}{API}{Application Programming Interface}
\newacronym[see={[Glossary:]{database-management-system}}]{DBMS}{DBMS}{Database Management System}
\newacronym[see={[Glossary:]{data-access-object}}]{DAO}{DAO}{data access object}
\newacronym[see={[Glossary:]{random-access-memory}}]{RAM}{RAM}{Random Access Memory}
\newacronym[see={[Glossary:]{control-version-system}}]{CVS}{CVS}{control version system}
\newacronym[see={[Glossary:]{end-to-end-test}}]{E2ET}{E2E test}{end-to-end test}
\newacronym[see={[Glossary:]{cascading-style-sheets}}]{CSS}{CSS}{Cascading Style Sheets}
\newacronym[see={[Glossary:]{document-object-model}}]{DOM}{DOM}{Document Object Model}
