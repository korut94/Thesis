%% TODO: Improve "conoscenze canoniche d'informatica" no clear, l.35
%% TODO: Replace "sottoparte" with something else, l.67
%% TODO: Crete reference with the "Prestazioni modulo NER" section, l. to define

\chapter{Il progetto di stage}
\label{cap:progetto-stage}

\section{Dalle recensione alle annotazioni}
Il compito per cui la Wonderflow viene commissionata è ricercare e riportare
statistiche di gradimento dei clienti su prodotti richiesti dai vari
committenti; siano rasoi elettrici, cellulari o altri prodotti elettronici e
non, il compito principale della Wonderflow è di ottenere queste informazioni.

Le preziosi fonti, dove l'azienda attinge, sono le recensioni dei prodotti
degli utenti nei più grandi servizi di vendita online (es. Amazon), in cui è
subito disponibile un loro cospicuo numero senza limiti e costi aggiunti
rendendo davvero semplice l'operazione di raccolta dati. La recensione
però è un'informazione grezza, non immediatamente utilizzabile per le seguenti
ragioni:
\begin{itemize}
\item E' in forma di testo libero
\item Contiene all'interno informazioni su più aspetti del prodotto
\item Spesso la descrizione non è chiara, a volte contraddittoria
\item Può essere scritta in più lingue
\end{itemize}

tutto ciò dovuto all'autore della recensione: l'umano. \\

A fronte di questo, la Wonderflow ha progettato un processo di raffinamento
(fig. \ref{fig:review_process}) per poter usufruire delle informazioni
all'interno delle recensioni, ed è proprio qui che entra in gioco in modo
preponderante la figura dell'\textbf{analista}.

\subsection{Descrizione del processo}
\label{subsec:processo_recensioni_annotazioni}
L'intera catena è divisa in 5 parti:
\begin{enumerate}
  \item Download recensioni (\textit{Developer})
  \item Distribuzione recensione agli analisti (\textit{Operations Manager})
  \item Analisi (\textit{Analysts})
  \item Controllo qualità (\textit{Senior Analyst})
  \item Archiviazione annotazioni approvate (\textit{Developer})
\end{enumerate}

\begin{figure}[ht]
\begin{center}
\includegraphics[height=2cm]{review_process}
\caption{Rappresentazione del processo da recensione ad annotazioni}
\label{fig:review_process}
\end{center}
\end{figure}

La primo trattato consiste nel \textit{download} delle recensioni da parte di
un dipendente, solitamente uno sviluppatore. Il motivo è che la procedura
richiede l'avvio di una serie di \gls{script} e controlli fuori dalle
competenze di personale con conoscenze canoniche d'informatica.
Una volta scaricate le recensioni dai vari siti web di vendita online, esse
vengono passate all'\textit{Operations Manager} in modo che possa distribuirle
tra i vari \textbf{analisti} in base alla disponibilità e alle abilità.

Una volta che l'analista ha a disposizione le recensioni da analizzare può
iniziare a cercare le \textbf{annotazioni}. Come detto precedentemente le
annotazioni sono testo in cui si può dedurre un \textbf{sentimento} su un
aspetto del prodotto.

Essendo che in una recensione l'autore può aver dato più commenti sulle varie
caratteristiche dell'articolo ci si aspetta di avere un'annotazione, o più, per
ogni commento. E' da precisare che il testo dell'annotazione non coincide quasi
mai con il testo del commento ma è ben si una sua sottoparte che possa
rappresentarlo al meglio.

La procedura per individuare annotazioni viene eseguita in un tool interno
chiamato \textbf{Admin Wonderflow} dove, tra i tanti servizi offerti, vi è
l'\textbf{editor} di selezione delle annotazioni. L'analista, una volta aperto
l'editor, è in grado di selezionare la recensione, visualizzarne il contenuto ed
iniziare l'analisi. Per creare un'annotazione inseribile nel sistema
informatico aziendale, l'analista deve necessariamente seguire la seguente
serie di passaggi:

\paragraph{Evidenziarla}
\label{evidenziarla}
L'analista, con il cursore del mouse, evidenzia la frazione di testo della
recensione dove ha individuato il commento.

\paragraph{Riassumerla}
\label{riassumerla}
L'analista deve riassumere il testo sottolineato individuando un periodo che
meglio rappresenti il commento. Nel caso l'annotazione fosse espressa in una
lingua diversa dall'inglese è necessario che l'analista fornisca sia la
traduzione che il testo originale.

\paragraph{Assegnargli un sentimento}
\label{sentimento}
E' compito dell'analista decidere se quell'annotazione ha un significato
\textbf{positivo} oppure \textbf{negativo}.

\paragraph{Catalogarla}
\label{catalogarla}
Le categorie nascono per raggruppare tutte quelle annotazioni che si
riferiscono ad una stessa caratteristica di un prodotto. L'analista deve
associare per ogni annotazione una delle categorie offerte dall'editor.

\paragraph{Intitolarla}
\label{intitolarla}
È richiesto che l'analista dia un titolo all'annotazione.

\paragraph{Salvarla}
\label{salvarla}
L'analista dichiara di aver terminato l'analisi di una recensione e sottomette
le annotazioni prodotte al controllo qualità. \\

Durante questa procedura verrà generata una struttura dati in formato \gls{JSON}
con il seguente schema (NdR: per motivi di leggibilità riporto solo il nome ed
il tipo di ogni attributo):

\begin{center}
\begin{lstlisting}[frame=single]
{
  "category": "string",
  "name": "string",
  "text" : "string",
  "syn" : "string",
  "sentiment" : "number",
  "start" : "number",
  "offset" : "number",
  "uid" : "string",
  "version" : "string",
  "origin" : "string",
  "id" : "string"
}
\end{lstlisting}
\end{center}

La tabella sottostante descrive il significato di ogni campo, il suo utilizzo e
in quale fase viene prodotto.

\begin{center}
\begin{longtable}{|>{\centering}p{2.2cm}|p{8cm}|>{\centering}p{2.5cm}|}
\hline
\textbf{Attributo} & \textbf{Descrizione} & \textbf{Fase} \tabularnewline \hline
\textit{category} &
Categoria di appartenenza di un'annotazione. Permette di capire su quale
aspetto di un prodotto l'annotazione offre un giudizio. &
\nameref{catalogarla} \tabularnewline \hline
\textit{name} &
Testo dell'annotazione e traduzione in inglese nel caso il testo originale
fosse espresso in un'altra lingua. Rappresenta l'annotazione vera e propria. &
\nameref{riassumerla} \tabularnewline \hline
\textit{text} &
Testo del commento evidenziato dall'analista. Il salvataggio del commento
risolve l'esigenza, soprattutto in fase di controllo sulla qualità, di avere il
contesto di dove l'annotazione è stata presa e quindi di poter dedurre se questa
è pertinente oppure no. &
\nameref{evidenziarla} \tabularnewline \hline
\textit{syn} &
Traduzione in lingua originale dell'annotazione. Nei sviluppi futuri potrà
essere usato per garantire l'identificazione di una medesima annotazione in
diverse lingue. &
\nameref{riassumerla} \tabularnewline \hline
\textit{sentiment} &
Il sentimento espresso dall'annotazione. Il sentimento positivo viene
rappresentato con l'intero 1, mentre il sentimento negativo con l'intero 0. &
\nameref{sentimento} \tabularnewline \hline
\textit{start} ed \textit{offset} &
Posizione di partenza del testo dell'annotazione e la sua lunghezza. &
\nameref{evidenziarla} \tabularnewline \hline
\textit{uid} &
Titolo dell'annotazione. &
\nameref{intitolarla} \tabularnewline \hline
\textit{version} &
Versione dell'annotazione. Disambigua le annotazioni prodotto dopo una
aggiornamento consistente sulla gestione delle annotazioni. Viene inizializzato
automaticamente con il valore ``v2'' dall'editor. &
\nameref{salvarla} \tabularnewline\hline
\textit{origin} &
Settore della recensione dov'è stata individuata l'annotazione all'interno della
recensione. I possibili valori sono:
\begin{itemize}
\item ``free-text''
\item ``pros''
\item ``cons''
\end{itemize}
nel caso l'annotazione fosse stata presa rispettivamente nel testo libero, nei
pregi oppure nei difetti &
\nameref{evidenziarla} \tabularnewline \hline
\caption{Descrizione attributi di un'annotazione in JSON con la relativa fase}
\label{tab:attributi_annotazione}
\end{longtable}
\end{center}

Una volta che l'annotazione è stata revisionata dal \textit{Senior Analyst},
risultando positiva ai test sulla qualità, può essere archiviata ed elaborata
dal sistema per poter fornire dati alla \textbf{Wonderboard}.

\subsection{Errori nel processo d'analisi}
Il processo d'analisi delle recensioni è di primaria importanza in quanto
fornisce i dati necessari per offrire i servizi proposti dall'azienda.
Tuttavia per poterlo eseguire, si è costretti a richiedere continuamente
l'intervento di operatori umani data la natura non automatizzabile delle
operazioni che si vanno a compiere.

Ne consegue che il sistema perde la scalabilità, aumenta i costi di mantenimento
ma soprattutto è necessaria un'attenta e costante supervisione del lavoro svolto
dagli analisti che può facilmente essere soggetto ad errori.

Durante il periodo dello stage è stato possibile comprendere i principali
fattori che conducono l'analista a commettere sbagli durante l'analisi delle
recensioni (i risultati sono riportati nella tabella
\ref{tab:fattori_errore}).

\begin{table}[ht]
\begin{center}
\begin{tabular}{|c|p{10cm}|}
\hline
\textbf{Fattore} & \textbf{Conseguenza} \\ \hline
Inesperienza &
L'analista non comprende a pieno la recensione creando annotazioni di poca
utilità. È inoltre possibile che non comprenda a pieno quali siano le
caratteristiche per produrre una buona annotazione, sfociando in risultati
scadenti. \\ \hline
Carico di lavoro &
L'analista, dove un certo numero di ore di lavoro, tende a perdere la
concentrazione aumentando la possibilità di commettere errori. \\ \hline
Tempistiche &
A volte l'analisi delle recensioni devono essere svolte in tempi più stretti.
Questo porta l'analista a svolgere un lavoro frettoloso.
\\ \hline
\end{tabular}
\end{center}
\caption{Fattori d'errore durante l'analisi delle recensioni}
\label{tab:fattori_errore}
\end{table}

\section{Le componenti sviluppate}
Ora che è stato introdotto il quadro generale per l'estrapolazione delle
annotazioni dentro le recensioni, è possibile parlare dell'obiettivo dello stage
tenuto presso la Wonderflow. Come è stato detto un grosso problema è l'analisi
delle recensioni tenuta dall'analista, la quale sovente commette una serie
d'errori che deficitano la produttività dell'azienda.

In questa sezione verranno esposi le varie componenti sviluppate nel corso dello
stage al fine di ridurre gli errori, garantendo un miglioramento dell'efficacia
nel processo di analisi delle recensioni.

\subsection{Il modulo NER}
Il progetto di stage mira al miglioramento del lavoro dell'analista introducendo
un supporto nell'individuazione delle possibile annotazioni. Più precisamente
si tratta di un modulo \gls{NER} grafico il quale individua nelle recensioni
annotazioni basandosi su quelle precedentemente archiviate.

Il suo uso è molto semplice: all'analista, entrando nell'editor, gli vengono
immediatamente suggerite delle possibili annotazioni nello stesso formato di
quelle individuate. Questo significa che l'analista si troverà nelle recensioni
le annotazioni già evidenziate e i campi dati richiesti già inseriti.

Ora sta all'analista decidere di scegliere se accettarle o meno, in quanto è
possibile che all'interno di quella recensione l'annotazione suggerita non sia
corretta. Nel caso l'accettasse, l'annotazione si inserirebbe nella lista di
quelle da validare; in caso non l'accettasse verrebbe semplicemente rimossa
dall'editor. \\

Per capire meglio la sua utilità è comodo paragonarlo ad uno strumento presente
da molto tempo nei sistemi di messaggistica per cellulari, quando l'utente è
intento a comporre le parole: il correttore. Mentre scrive la parola,
all'utente gli vengono visualizzate delle possibili scelte che può usare.
Selezionando la scelta desiderata si vedrà la porzione di parola scritta essere
sostituita con la parola suggerita. Questo ancora oggi è uno strumento di
notevole aiuto che aiuta a velocizzare la scrittura dei messaggi e ad evitare di
commettere errori di battitura.

Analogamente per le annotazioni, ora l'analista non deve più seguire tutti i
passaggi che portano alla creazione delle annotazioni (descritto nella
sottosezione \ref{subsec:processo_recensioni_annotazioni}), ma è sufficiente
confermare il suggerimento dato dal sistema \gls{NER}. Oltre a risparmiare
tempo, esattamente come il correttore automatico, si evitano tutti possibili
errori che si possono incorrere in questa fase per i fattori esposti nella
tabella \ref{tab:fattori_errore}. Inoltre, in fase di controllo della qualità le
annotazioni create attraverso il sistema di riconoscimento vengono approvate
automaticamente essendo prese d'annotazioni precedentemente controllate;
riducendo anche il tempo del \textit{Senior Analyst} per il controllo qualità.

\subsection{Moderazione del glossario}
