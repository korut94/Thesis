%% TODO: Improve "conoscenze canoniche d'informatica" no clear, l.35
%% TODO: Replace "sottoparte" with something else, l.67

\chapter{Il progetto di stage}
\label{cap:progetto-stage}

\section{Errare è umano}
Il compito per cui la Wonderflow viene commissionata è ricercare e riportare
statistiche di gradimento dei clienti su prodotti richiesti dai vari
committenti; siano rasoi elettrici, cellulari o altri prodotti elettronici e
non, il compito principale della Wonderflow è di ottenere queste informazioni.

Le preziosi fonti, dove l'azienda attinge, sono le recensioni dei prodotti
degli utenti nei più grandi servizi di vendita online (es. Amazon), in cui è
subito disponibile un loro cospicuo numero senza limiti e costi aggiunti
rendendo davvero semplice l'operazione di raccolta dati. La recensione
però è un'informazione grezza, non immediatamente utilizzabile per le seguenti
ragioni:
\begin{itemize}
\item E' in forma di testo libero
\item Contiene all'interno informazioni su più aspetti del prodotto
\item Spesso la descrizione non è chiara, a volte contraddittoria
\item Può essere scritta in più lingue
\end{itemize}

tutto ciò dovuto all'autore della recensione: l'umano. \\

A fronte di questo, la Wonderflow ha progettato un processo di raffinamento
(fig. \ref{fig:review_process}) per poter usufruire delle informazioni
all'interno delle recensioni, ed è proprio qui che entra in gioco in modo
preponderante la figura dell'\textbf{analista}.

\subsection{Descrizione del processo}
L'intera catena è divisa in 5 parti:
\begin{enumerate}
  \item Download recensioni (\textit{Developer})
  \item Distribuzione recensione agli analisti (\textit{Operations Manager})
  \item Analisi (\textit{Analysts})
  \item Controllo qualità (\textit{Senior Analyst})
  \item Archiviazione annotazioni approvate (\textit{Developer})
\end{enumerate}

\begin{figure}[ht]
\begin{center}
\includegraphics[height=2cm]{review_process}
\caption{Rappresentazione del processo da recensione ad annotazioni}
\label{fig:review_process}
\end{center}
\end{figure}

La primo trattato consiste nel \textit{download} delle recensioni da parte di
un dipendente, solitamente uno sviluppatore. Il motivo è che la procedura
richiede l'avvio di una serie di \gls{script} e controlli fuori dalle
competenze di personale con conoscenze canoniche d'informatica.
Una volta scaricate le recensioni dai vari siti web di vendita online, esse
vengono passate all'\textit{Operations Manager} in modo che possa distribuirle
tra i vari \textbf{analisti} in base alla disponibilità e alle abilità.

Una volta che l'analista ha a disposizione le recensioni da analizzare può
iniziare a cercare le \textbf{annotazioni}. Come detto precedentemente le
annotazioni sono testo in cui si può dedurre un \textbf{sentimento} su un
aspetto del prodotto.

Essendo che in una recensione l'autore può aver dato più commenti sulle varie
caratteristiche dell'articolo ci si aspetta di avere un'annotazione, o più, per
ogni commento. E' da precisare che il testo dell'annotazione non coincide quasi
mai con il testo del commento ma è ben si una sua sottoparte che possa
rappresentarlo al meglio.

La procedura per individuare annotazioni viene eseguita in un tool interno
chiamato \textbf{Admin Wonderflow} dove, tra i tanti servizi offerti, vi è
l'\textbf{editor} di selezione delle annotazioni. L'analista, una volta aperto
l'editor, è in grado di selezionare la recensione, visualizzarne il contenuto ed
iniziare l'analisi. Per creare un'annotazione inseribile nel sistema
informatico aziendale, l'analista deve necessariamente seguire la seguente
serie di passaggi:

\paragraph{Evidenziarla}
\label{evidenziarla}
L'analista, con il cursore del mouse, evidenzia la frazione di testo della
recensione dove ha individuato il commento.

\paragraph{Riassumerla}
\label{riassumerla}
L'analista deve riassumere il testo sottolineato individuando un periodo che
meglio rappresenti il commento. Nel caso l'annotazione fosse espressa in una
lingua diversa dall'inglese è necessario che l'analista fornisca sia la
traduzione che il testo originale.

\paragraph{Assegnargli un sentimento}
\label{sentimento}
E' compito dell'analista decidere se quell'annotazione ha un significato
\textbf{positivo} oppure \textbf{negativo}.

\paragraph{Catalogarla}
\label{catalogarla}
Le categorie nascono per raggruppare tutte quelle annotazioni che si
riferiscono ad una stessa caratteristica di un prodotto. L'analista deve
associare per ogni annotazione una delle categorie offerte dall'editor.

\paragraph{Intitolarla}
\label{intitolarla}
È richiesto che l'analista dia un titolo all'annotazione.

\paragraph{Salvarla}
\label{salvarla}
L'analista dichiara di aver terminato l'analisi di una recensione e sottomette
le annotazioni prodotte al controllo qualità. \\

Durante questa procedura verrà generata una struttura dati in formato \gls{JSON}
con il seguente schema (NdR: per motivi di leggibilità riporto solo il nome ed
il tipo di ogni attributo):

\begin{center}
\begin{lstlisting}[frame=single]
{
  "category": "string",
  "name": "string",
  "text" : "string",
  "syn" : "string",
  "sentiment" : "number",
  "start" : "number",
  "offset" : "number",
  "uid" : "string",
  "version" : "string",
  "origin" : "string",
  "id" : "string"
}
\end{lstlisting}
\end{center}

La tabella sottostante descrive il significato di ogni campo, il suo utilizzo e
in quale fase viene prodotto.

\begin{center}
\begin{longtable}{|>{\centering}p{2.2cm}|p{8cm}|>{\centering}p{2.5cm}|}
\hline
\textbf{Attributo} & \textbf{Descrizione} & \textbf{Fase} \tabularnewline \hline
\textit{category} &
Categoria di appartenenza di un'annotazione. Permette di capire su quale
aspetto di un prodotto l'annotazione offre un giudizio. &
\nameref{catalogarla} \tabularnewline \hline
\textit{name} &
Testo dell'annotazione e traduzione in inglese nel caso il testo originale
fosse espresso in un'altra lingua. Rappresenta l'annotazione vera e propria. &
\nameref{riassumerla} \tabularnewline \hline
\textit{text} &
Testo del commento evidenziato dall'analista. Il salvataggio del commento
risolve l'esigenza, soprattutto in fase di controllo sulla qualità, di avere il
contesto di dove l'annotazione è stata presa e quindi di poter dedurre se questa
è pertinente oppure no. &
\nameref{evidenziarla} \tabularnewline \hline
\textit{syn} &
Traduzione in lingua originale dell'annotazione. &
\nameref{riassumerla} \tabularnewline \hline
\textit{sentiment} &
Il sentimento espresso dall'annotazione. Il sentimento positivo viene
rappresentato con l'intero 1, mentre il sentimento negativo con l'intero 0. &
\nameref{sentimento} \tabularnewline \hline
\textit{start} ed \textit{offset} &
Posizione di partenza del testo dell'annotazione e la sua lunghezza. &
\nameref{evidenziarla} \tabularnewline \hline
\textit{uid} &
Titolo dell'annotazione. &
\nameref{intitolarla} \tabularnewline \hline
\textit{version} &
Versione dell'annotazione. Disambigua le annotazioni prodotto dopo una
aggiornamento consistente sulla gestione delle annotazioni. Viene inizializzato
automaticamente con il valore ``v2'' dall'editor. &
\nameref{salvarla} \tabularnewline\hline
\textit{origin} &
Settore della recensione dov'è stata individuata l'annotazione all'interno della
recensione. I possibili valori sono:
\begin{itemize}
\item free-text
\item pros
\item cons
\end{itemize}
nel caso l'annotazione fosse stata presa rispettivamente nel testo libero, nei
pregi oppure nei difetti &
\nameref{evidenziarla} \tabularnewline \hline
\caption{Descrizione attributi di un'annotazione in JSON con la relativa fase}
\end{longtable}
\end{center}
