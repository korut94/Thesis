%% TODO: Improve "conoscenze canoniche d'informatica" no clear, l.35

\chapter{Il progetto di stage}
\label{cap:progetto-stage}

\section{Errare è umano}
Il compito per cui la Wonderflow viene commissionata è ricercare e riportare
statistiche di gradimento dei clienti su prodotti richiesti dai vari
committenti; siano rasoi elettrici, cellulari o altri prodotti elettronici e
non, il compito principale della Wonderflow è di ottenere queste informazioni.

Le preziosi fonti, dove l'azienda attinge, sono le recensioni dei prodotti
degli utenti nei più grandi servizi di vendita online (es. Amazon), in cui è
subito disponibile un loro cospicuo numero senza limiti e costi aggiunti
rendendo davvero semplice l'operazione di raccolta dati. La recensione
però è un'informazione grezza, non immediatamente utilizzabile per le seguenti
ragioni:
\begin{itemize}
\item E' in forma di testo libero
\item Contiene all'interno informazioni su più aspetti del prodotto
\item Spesso la descrizione non è chiara, a volte contraddittoria
\item Può essere scritta in più lingue
\end{itemize}

tutto ciò dovuto all'autore della recensione: l'umano. \\

A fronte di questo, la Wonderflow ha progettato un processo di raffinamento per
poter usufruire delle informazioni all'interno delle recensioni, ed è proprio
qui che entra in gioco in modo preponderante la figura dell'\textbf{analista}.

\subsection{Descrizione del processo}
La primo trattato consiste nel \textit{download} delle recensioni da parte di
un dipendente, solitamente uno sviluppatore. Il motivo è che la procedura
richiede l'avvio di una serie di \gls{script} e controlli fuori dalle
competenze di personale con conoscenze canoniche d'informatica.
Una volta scaricate le recensioni dai vari siti web di vendita online, esse
vengono passate all'\textit{Operations Manager} in modo che possa distribuirle
tra i vari \textbf{analisti} in base alla disponibilità e alle abilità.

Una volta che l'analista ha a disposizione le recensioni da analizzare può
iniziare a cercare le \textbf{annotazioni}. Come detto precedentemente le
annotazioni sono testo in cui ne si può dedurre un \textbf{sentimento} su un
aspetto del prodotto, da cui è possibile alimentare la \textbf{Wonderboard} con
tutta una serie di statistiche interessanti per il committente.
Essendo che in una recensione l'autore può aver dato più commenti sulle varie
caratteristiche dell'articolo, ci si aspetta di avere un'annotazione o più per
ogni commento.

La procedura per individuare annotazioni viene eseguita in un tool interno
chiamato \textbf{Admin Wonderflow} dove, tra i tanti servizi offerti, vi è
l'\textbf{editor} di selezione delle annotazioni. All'analista, una volta aperto
l'editor, è in grado di selezionare la recensione, visualizzarne il contenuto ed
iniziare l'analisi. Per creare l'annotazione utilizzabile dal è sufficiente
selezionare il testo interessato in modo che l'editor esegua in automatico la
procedura d'estrazione del contenuto. 

\begin{figure}[ht]
\begin{center}
\includegraphics[height=2cm]{review_process}
\caption{Rappresentazione del processo da recensione ad annotazioni}
\label{fig:review_process}
\end{center}
\end{figure}
