\chapter{Conclusioni}
\label{cap:conclusioni}
\section{Raggiungimento degli obiettivi}


\section{Bilancio formativo}
Durante il periodo dello stage sono state svolte un considerevole numero d'
attività, andando a coprire quasi tutto lo spettro di tematiche dietro la
realizzazione di un'applicazione web. Partendo dal \gls{back-end}: manutenzione
del \gls{BaaS}, interfacciamento dei \textit{database MongoDB}, realizzazione di
script per vari scopi; fino ad arrivare alla parte \gls{front-end}: 
progettazione per componenti offerta dal \textit{framework AngularJS} e
comunicazione in un'architettura \gls{REST}. Inoltre, la esigenze di sviluppo
hanno permesso di scontrarsi con una ragguardevole varietà di argomenti legati
all'\textit{Ingegneria del Software}: architettura del sistema e del prodotto, 
\textit{Design Pattern}, verifica, validazione e fasi, come il rilascio
e la manuntezione del software, esclusi dal progetto didattico; accompagnati
da tecnologie e pratiche, per supportare, che vanno ad aggiungersi al personale
bagaglio culturale.

Tuttavia, ritengo che sia da tutt'altra parte il maggior beneficio dell'
esperienza. L'aforisma di \textit{Thomas Hobbes} ``\textit{Non imparare dai tuoi
errori. Impara dagli errori degli altri cosi che tu possa non farne}'' 
rappresenta perfettamente ciò che è accaduto duranto lo stage. Le difficoltà 
riscontrate durante i lavori, o l'adozzione di pratiche sbagliate erano molto 
spesso l'effetto di una mancata presenza di sistemi come la \textit{Continuos
Integration} o una progettazione superficiale sul software. Le varie dissidie
però hanno portato una maggion comprensione e appropriazione delle conoscenze
acquisite nei tre anni di università.







